\documentclass[12pt]{article}
\usepackage[utf8]{inputenc}
\usepackage[english]{babel}
\usepackage[margin=1in]{geometry}
\usepackage{graphicx}
\usepackage{hyperref}
\usepackage{amsmath}
\usepackage{amsfonts}
\hypersetup{
    colorlinks,
    citecolor=black,
    filecolor=black,
    linkcolor=black,
    urlcolor=black
}

\title{MAT292 Notes}
\date{September 2021}

\begin{document}

\maketitle
\tableofcontents

\section{Existence and Uniqueness Theorem}

\begin{enumerate}
    \item We need $f(t,y)$ continuous in the rectangle to get existence
    \item We need $f_y(t,y)$ continuous in the rectangle to get uniqueness
    \item E \& U Theorem is sufficient but not necessary. i.e. these conditions imply solution but not having these conditions doesnt mean there is no solution
\end{enumerate}

\section{Autonomous Equations and Population Dynamics}

\subsection{Logistic Growth}

If uninhibited, we assume exp. growth however in the long run,population is limited to $K$

Model: $y' = r h(y) y$

We want $h(y) \approx 1$ if $y$ is small, $h(y) < 1$ if $y < k$, $h(y) = 0 $ if $y=k$ and $h(y) < 0$ if $y>K$ 

This can thus be modelled as $y' = r(1 - \frac{y}{k})y$. This has two equilibria namely at $y=0$ and $yk$. The inflection points can be found by setting the derivative $y''$ to 0.

\section{Direction Fields and Orbits}

\subsection{Reducing non homogeneous systems to homogeneous systems}

Lets take a solution $x$ and write it as $x = \phi + v$ where $v$ is a constant. Then $x' = Ax + b \rightarrow \phi = A(\phi + v) + b$. Since $x_{eq} = A^{-1}b$, $Av+b =0$ by the equilibrium condition ($\phi' - A \phi$) we have that $\phi' = A \phi$. So that $x = \phi + x_{eq}$ where $\phi$ is a solution of the homogeneous system. 

Every solution of the non homogeneous problem can be written as a solution of the homogeneous problem plus the equilibrium. 

\section{Laplace Transform}

\begin{itemize}
    \item Remark: The laplace transform will allow us to reduce solving an ODE to solving an algebraic equation
    \item Solve algebraic equation and use the inverse laplace transform to get the solution to the ODE
    \item Definition: If $f$ is defined on $[0, \infty]$, the Laplace Transform is defined as $F(s) = \int_{0}^{\infty} e^{-st} f(t) \, dt$
    \item We write $F = \mathcal{L} \{f\}$
    \item We use uppercase letters for Laplace transform e.g. $G(s)$ is the LT of $g(t)$
    \item Example: For $f(t) = e^{at}$, we get $F(s) = \mathcal{L}\{f\}(s) = \int_{0}^{\infty} e^{-st} e^{at} \, dt = \lim_{b\rightarrow \infty} \int_{0}^{b} e^{(a-s)t} \, dt = \lim_{b\rightarrow \infty} \frac{1}{a-s} \left( e^{(a-s)b} - 1 \right) = \frac{1}{s-a}$ if $s > a$
    \item $\mathcal{L} \{1\} = \frac{1}{s}$
    \item Theorem: $\mathcal{L} \{c_1f_1 + c_2f_2\} = c_1 \mathcal{L} \{f_1\} + c_2 \mathcal{L} \{f_2\}$
    \item To find $\mathcal{L} \{\sin(at)\}$, write $\sin(at) = \frac{1}{2i} (e^{iat} - e^{-iat})$ and use the theorem above
    \item This will give $\frac{1}{2i} \left(\frac{1}{s - ia}\right) - \frac{1}{2i} \left(\frac{1}{s+ia}\right) = \frac{a}{s^2 + a^2}$ for $s > 0$
    \item Example: LT of $f(t) = e^{2t}$ for $0 \leq t < 1$ and $f(t) = 4$ for $1 \leq t$
    \item Divide the integral into two seperate parts and evaluate it
    \item Exponential order: A function $f(t)$ is of exponential order for $M > 0$, $K > 0$ and $a \in \mathbb{R} $ if $|f(t)| \leq Ke^{at}$ for $t \geq M$ i.e. $f$ eventually becomes between two exponential functions
    \item Theorem: Every bounded function is of exponential order
    \item A function $f(t)$ is piecewise continuous on $[a,b]$ iff there are finitely many "jump points" between $a$ and $b$ $a \leq t_0 < t_1 < \dots < t_{k-1} < t_k = b$ such that $f$ is continuous on each of the intervals $(t_i, t_{i+1})$ and $f$ has finite limits at the jump points.
    \item Theorem: If for a function $f(t)$, we have that $f$ is piecewise continuous on $[0, A]$ $\forall A \geq 0$ and $f$ is of exponential order for $M, k$ and $a$. Then $\mathcal{L} \{f\}$ exists for all $s > a$. 
    \item Theorem: If $f(t)$ is of exponential order then we have: $F(s) \rightarrow 0$ as $s \rightarrow \infty$ where $F(s)$ is the L.T. of $f$
    \item Theorem: If $f$ is continuous and $f'$ is piecewise continuous on any interval $[0, A]$ and $f, f'$ are of exponential order for $M, k, a$ then $\mathcal{L} \{f'\} (s) = s \mathcal{L} \{f\} (s) - f(0)$ for $s>a$. Under the same conditions for $n$ derivatives, $\mathcal{L} \{f^{(n)}\} (s) = s^n \mathcal{L} \{f\} (s) - s^{n-1} f(0) - s^{n-2} f'(0) - \dots sf^{(n-2)} (0) - f^{(n-1)} (0)$ 
    \item Proof: $\mathcal{L} \{f'\} (s) = \int_{0}^{\infty} e^{-st} f'(t) \, dt = \lim_{b \rightarrow \infty} \left(\int_{0}^{b} e^{-st} f'(t) \, dt \right) $
    
    $= \lim_{b \rightarrow \infty} \left( \left[ e^{-st} f(t) \right]_{0}^{b} + \int_{0}^{b} f(t) s e^{-st} \, dt \right) = \lim_{b \rightarrow \infty} \left( e^{-bs}f(b) - f(0) + s \int_{0}^{b} f(t) e^{-st} \, dt \right) $ 
    $= s \mathcal{L} \{f\} (s) - f(0)$ where $s > a$ (by definition of exponential order)
\end{itemize} 

\section{Inverse Laplace Transform}

\begin{itemize}
    \item Theorem: If $f(t), g(t)$ are piecewise continuous and of exponential order, then $\mathcal{L} \{f\} = \mathcal{L} \{g\} \implies f(t) = g(t)$
    \item Technicality: Take $f(t) = e^t$, $g(t) = \begin{cases} 
        e^t & t \neq 5 \\
        0 & t = 5  
     \end{cases}$. Clearly $\mathcal{L} \{f\} = \mathcal{L} \{g\}$ but $f(t) \neq g(t) \, \forall t$
     \item Convention: We write $f(t) = g(t)$ as long as they are the same whenever they are continuous
     \item Definition: If $f$ is piecewise continuous and of exponential order and $\mathcal{L} \{f\} (s) = F(s)$, then we call $f(t) = \mathcal{L}^{-1} \{F\} (t)$
     \item There is a complex analysis formula (Mellin Transform) to find $\mathcal{L}^{-1} \{F\}$. However this is rarely used in practice and we instead use tables
\end{itemize}

\section{Solving ODEs with Laplace Transform}
\begin{itemize}
    \item Lets solve the IVP $y'' + 2y' + 5y = e^{-t}$, $y(0) = 1$, $y'(0) = -3$ \begin{itemize}
        \item Use the Laplace transform: $\mathcal{L} \{y'' + 2y' + 5y\} = \mathcal{L} \{e^{-t} \}$
        \item $s^2 Y(s) - sy(0) - y'(0) + 2(sY(s) - y(0)) + 5Y(s) = \frac{1}{s+1}$
        \item $s^2 Y(s) - s \cdot 1 - (-3) + 2(sY(s) - 1) + 5 Y(s) = \frac{1}{s+1}$ using the initial conditions
        \item Solving this gives $Y(s) = \frac{s^2}{(s+1)(s^2+2s+5)}$
        \item Simplifying and using partial fractions, $Y(s) = \frac{1}{4} \frac{1}{s+1} + \frac{3}{4} \frac{s+1}{(s+1)^2 + 4} - \frac{2}{(s+1)^2 + 4}$
        \item $y(t) = \mathcal{L} ^{-1} \{Y(s) \} = \frac{1}{4} \mathcal{L} ^{-1} \{\frac{1}{s+1}\} + \frac{3}{4} \mathcal{L} ^{-1} \{\frac{s+1}{(s+1)^2+4}\} - 2 \mathcal{L}^{-1} \{ \frac{1}{(s+1)^2+4}\}$ 
        \item $ = \frac{1}{4} e^{-t} + \frac{3}{4} e^{-t} \cos (2t) - e^{-t} \sin (2t)$
    \end{itemize}
    \item Lets solve the IVP $y'''' + 2y' + y = 0$, $y(0) = 1$, $y'(0) = -1$, $y''(0) = 0$, $y'''(0) = 2$ \begin{itemize}
        \item Applying the Laplace transform: $s^4 Y(s) - s^3 y(0) - s^2 y'(0) - s y''(0) - y'''(0) + 2 (s^2 Y(s) - sy(0) - y'(0)) + Y(s) = 0$
        \item Using the initial conditions, $s^4 Y(s) - s^3 + s^2 - 2 + 2(s^2Y(s) - s + 1) + Y(s) = 0$
        \item Solving gives $Y(s) = \frac{s^3 - s^2 + 2s}{(s^2+1)^2}$
        \item We use a repeated partial fraction decomposition to write $Y(s)$ as $\frac{As+B}{s^2+1} + \frac{s+1}{(s^2+1)^2}$
        \item We know $y(t) = \mathcal{L} ^{-1} \{ \frac{s}{s^2+1} - \frac{1}{s^2+1} + \frac{s}{(s^2+1)^2} + \frac{1}{(s^2+1)^2} \}$
        \item For the third term, we use $\mathcal{L} \{t \cdot f(t) \} = \frac{-d}{ds} F(s)$ and then get $ y(t) = \cos t - \sin t + \frac{1}{2} t \sin t + \frac{1}{2} \sin t - \frac{1}{2} t \cos t $
    \end{itemize}
\end{itemize}

\section{Discontinuous Forcing Functions}

\begin{itemize}
    \item Finding $\mathcal{L} \{ t \} = \int_{0}^{\infty} e^{-st} t \, dt = \frac{1}{s^2}$ (IBP) or using $\mathcal{L} \{t\} = \mathcal{L} \{t \cdot 1 \}$ and $\mathcal{L} \{ t \cdot 1 \} = - \frac{d}{ds} (\frac{1}{s}) = \frac{1}{s^2}$
    \item Consider $y'' + 4y = g(t)$ with $y(0) = 0$, $y'(0) = 0$ and $g(t) = \begin{cases} 
        0 & 0 < t < 5 \\
        \frac{t-5}{5} & t = 5 \leq t < 10 \\
        t & 10 \leq t
     \end{cases}$ \begin{itemize}
         \item Can split it up into 3 parts and find relevant boundary conditions in each case to be used in the next case 
     \end{itemize}
     \item Piecewise defined functions can simulate the activation of a signal \begin{itemize}
         \item $u_c(t)$ is a step function that is 0 till $x = c$ and then 1 afterwards
         \item $u_{cd} (t)$ is an indicator function which is 1 between $c$ and $d$ and 0 everywhere else
         \item Use a combination of step and indicator functions $= u_{cd} (t) \frac{t-c}{d-c} + u_d(t)$ which is 0 less than $c$, increasing between $c$ and $d$ and 1 for values greater than $d$
     \end{itemize}
     \item We can find the Laplace transforms of these functions \begin{itemize}
         \item $\mathcal{L} \{ u_c \} = \int_{0}^{\infty} e^{-st} u_c(t) \, dt = \int_{c}^{\infty} e^{-st} \cdot 1 \, dt = \frac{e^{-cs}}{s}$
         \item $\mathcal{L} \{ u_{cd} \} = \mathcal{L} \{ u_c - u_d \} = \frac{e^{-cs}}{s} - \frac{e^{-ds}}{s}$
     \end{itemize}
     \item The function $g(t)$ from earlier can be written as $u_{5,10} (t) \cdot \frac{t-5}{5} + u_{10} (t) \cdot 1 $ \begin{itemize}
         \item Rearrange this as $\frac{1}{5} \left[ u_{5,10} (t) (t-5) + 5 u_{10} \right]  = \frac{1}{5} \left[ (u_5(t) - u_{10}(t))(t-5) + 5 u_{10} (t) \right] = \frac{1}{5} \left[ u_5(t) (t-5) - u_{10} (t) (t-10) \right]$
         \item $h(t) = u_5(t) (t-5)$ is the time shift of $f$ by $t = 5$
     \end{itemize}
     \item Theorem (Laplace transform of time-shift): If $F = \mathcal{L} \{f\}$ exists for $s > a$ and $c \geq 0$, then $\mathcal{L} \{ u_c(t) f(t-c) \} = \int_{0}^{\infty} e^{-st} u_c(t) f(t-c) \, dt = \int_{c}^{\infty} e^{-st} f(t-c) \, dt = \int_{0}^{\infty} e^{-s(u+c)} f(u) \, du = e^{-sc} \int_{0}^{\infty} e^{-su} f(u) \, du = e^{-sc} \mathcal{L} \{f\}$
\end{itemize}


\end{document}
